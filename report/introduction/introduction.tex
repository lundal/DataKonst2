\subsection{Pipelining}

Pipelining is a technique to increase the throughput of the processor. Multiple
steps in the processor can be performed at the same time. 

\begin{figure}[ht]
    \centering
    \includegraphics[scale=0.7]{figures/pipeline.png}
    \caption{\label{fig:pipeline}An example of pipeline during execution} 
\end{figure}

Pipelined design is especialy good when the processor is multi-cycle, as the
instructions are divided into smaller parts. In other words, pipelining is a way
of making a multi-cycle processor even faster. 

\subsubsection*{Hazards}
But pipelining is not only good, as it can cause problems when the instructions
needs to perform the same action at different times, or some data is dependent
on an earlier instruction. This can always be solved by stalling the processor, 
but it can't be done while running. To stop this, the processor must do so called 
hazard detection. When it finds that a problem will arise, it adds 'stall 
instructions', or pipeline bubbles, that basically tells the processor to stall 
for that cycle. In other words, it inserts a NOP instruction into the pipeline.

\subsection{Optimization}

\subsubsection*{Forwarding}
Instead of inserting pipeline bubbles, forwarding can be implemented. To forward
is to send the result of an instruction back in the pipeline to an earlier
dependent step. A trivial example of this can be

\begin{verbatim}
a = 1 + 1
b = a + 1
\end{verbatim}

as \textit{b} is dependent on \textit{a}, and that instruction needs to be done.
By sending \textit{a} back, the second instruction can be completed in time.


\subsubsection*{Branch prediction}
In the case of branching, the processor can't add the next instruction to the
pipeline, as it isn't sure of what instruction is the next. To solve this
problem, branch prediction tries to guess whether or not a branch will be
followed. This allows to then find the next instruction, and thus add the next
instruction to the pipeline.


\subsubsection*{Flushing}
If a branch is taken, but was assumed not, the instruction in the pipeline must
be discarded. This is called flushing, and is done by inserting NOP instructions
instead of instructions that should be executed.
