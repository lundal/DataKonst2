\documentclass[a4paper]{article}
\usepackage[parfill]{parskip} % For line skip paragraphs
\usepackage[utf8]{inputenc}
\usepackage{geometry}
\usepackage{fancyhdr}
\usepackage[pdftex]{graphicx}
\usepackage{cite}
\usepackage{listings}
\usepackage{caption}
\usepackage{color}
\usepackage{underscore}
\usepackage{url}

\usepackage{courier}

% from http://stackoverflow.com/questions/741985/latex-source-code-listing-like-in-professional-books
 \lstset{
         basicstyle=\footnotesize\ttfamily, % Standardschrift
         numbers=left,               % Ort der Zeilennummern
         numberstyle=\tiny,          % Stil der Zeilennummern
         %stepnumber=2,               % Abstand zwischen den Zeilennummern
         numbersep=5pt,              % Abstand der Nummern zum Text
         tabsize=2,                  % Groesse von Tabs
         extendedchars=true,         %
         breaklines=true,            % Zeilen werden Umgebrochen
         keywordstyle=\color{red},
            frame=b,         
%        keywordstyle=[1]\textbf,    % Stil der Keywords
%        keywordstyle=[2]\textbf,    %
%        keywordstyle=[3]\textbf,    %
%        keywordstyle=[4]\textbf,   \sqrt{\sqrt{}} %
         stringstyle=\color{white}\ttfamily, % Farbe der String
         showspaces=false,           % Leerzeichen anzeigen ?
         showtabs=false,             % Tabs anzeigen ?
         xleftmargin=17pt,
         framexleftmargin=17pt,
         framexrightmargin=5pt,
         framexbottommargin=4pt,
%         showstringspaces=false      % Leerzeichen in Strings anzeigen ?        
 }

\DeclareCaptionFont{white}{ \color{white} }
\DeclareCaptionFormat{listing}{
  \colorbox[cmyk]{0.43, 0.35, 0.35,0.01 }{
    \parbox{\textwidth}{\hspace{15pt}#1#2#3}
  }
}
\captionsetup[lstlisting]{ format=listing, labelfont=white, textfont=white, singlelinecheck=false, margin=0pt, font={bf,footnotesize} }

%TCIDATA{OutputFilter=LATEX.DLL}
%TCIDATA{Version=5.50.0.2953}
%TCIDATA{<META NAME="SaveForMode" CONTENT="1">}
%TCIDATA{BibliographyScheme=Manual}
%TCIDATA{Created=Monday, January 30, 2012 17:20:46}
%TCIDATA{LastRevised=Monday, February 27, 2012 12:06:08}
%TCIDATA{<META NAME="GraphicsSave" CONTENT="32">}
%TCIDATA{<META NAME="DocumentShell" CONTENT="Standard LaTeX\Blank - Standard LaTeX Article">}
%TCIDATA{CSTFile=40 LaTeX article.cst}

\newenvironment{proof}[1][Proof]{\noindent\textbf{#1.} }{\ \rule{0.5em}{0.5em}}

% Sets page margins to 1", which is standard
\geometry{left=1in,right=1in,top=1in,bottom=1in} 

% allows the included extensions of graphic files
\DeclareGraphicsExtensions{.pdf,.png,.jpg}

% sets/adds graphic path. If empty it just looks around the folder the .tex file is in
\graphicspath{{}}

% I do not remember what this does
\setlength{\headheight}{15.2pt}

% allows the xhead parameters
\pagestyle{fancy}

% Sets the left header
\lhead{Bye, Gombos \& Lundal}

% Sets the right header
\rhead{Exercise 2, TDT4255 Autumn 2013}

% everything before this is considered the header or whatever.
\begin{document}

% INCLUDEGRAPHICS EXPLANATION
% \includegraphics[scale=1]{name of file}
% sometimes you want to twice encase the filename in squiggly brackets. I do not know why but sometimes it is required.

% begin title page, use \\ for newline
\title{Report for Exercise 2\\Optimized Pipelined MIPS Processor\\\vspace{2mm}\Large{TDT4255 Computer Design}}

% now one can list the authors, \textbf{} makes bold text
\author{Emil Taylor Bye \and Péter Henrik Gombos \and Per Thomas Lundal}


\pagenumbering{roman}

% make title page
\maketitle


\bigskip
\bigskip
\bigskip
\bigskip

\part*{Abstract}

This report describes the implementation of an optimized pipelined processor
with the MIPS instruction set. The processor is tested both on a computer and an
FPGA with programs written in assembly, and shown to work well.


\clearpage

\tableofcontents

\setcounter{secnumdepth}{3}

\clearpage

\setcounter{page}{1}
\pagenumbering{arabic}

\part{Introduction}

\subsection{Pipelining}

Pipelining is a technique to increase the throughput of the processor. Multiple
steps in the processor can be performed at the same time. 

\begin{figure}[ht]
    \centering
    \includegraphics[scale=0.7]{figures/pipeline.png}
    \caption{\label{fig:pipeline}An example of pipeline during execution} 
\end{figure}

Pipelined design is especialy good when the processor is multi-cycle, as the
instructions are divided into smaller parts. In other words, pipelining is a way
of making a multi-cycle processor even faster. 

\subsubsection*{Hazards}
But pipelining is not only good, as it can cause problems when the instructions
needs to perform the same action at different times, or some data is dependent
on an earlier instruction. This can always be solved by stalling the processor, 
but it can't be done while running. To stop this, the processor must do so called 
hazard detection. When it finds that a problem will arise, it adds 'stall 
instructions', or pipeline bubbles, that basically tells the processor to stall 
for that cycle. In other words, it inserts a NOP instruction into the pipeline.

\subsection{Optimization}

\subsubsection*{Forwarding}
Instead of inserting pipeline bubbles, forwarding can be implemented. To forward
is to send the result of an instruction back in the pipeline to an earlier
dependent step. A trivial example of this can be

\begin{verbatim}
a = 1 + 1
b = a + 1
\end{verbatim}

as \textit{b} is dependent on \textit{a}, and that instruction needs to be done.
By sending \textit{a} back, the second instruction can be completed in time.


\subsubsection*{Branch prediction}
In the case of branching, the processor can't add the next instruction to the
pipeline, as it isn't sure of what instruction is the next. To solve this
problem, branch prediction tries to guess whether or not a branch will be
followed. This allows to then find the next instruction, and thus add the next
instruction to the pipeline.


\subsubsection*{Flushing}
If a branch is taken, but was assumed not, the instruction in the pipeline must
be discarded. This is called flushing, and is done by inserting NOP instructions
instead of instructions that should be executed.


\clearpage

\part{Description}


\begin{figure}[ht]
    \centering
    \includegraphics[width=\textwidth]{figures/SuggestedArchitecture.png}
    \caption{The suggested CPU architecture in the compendium \cite[p.118]{lab-compendium}} 
    \label{fig:suggestedArchitecture}
\end{figure}



\clearpage

\part{Solution}

\section{Architecture}

\begin{figure}[ht]
    \centering
    \includegraphics[scale=0.11]{figures/Architecture.png}
    \caption{\label{fig:cpuArchitecture}The implemented CPU architecture} 
\end{figure}

\section{Instruction Set}

\section{Forwarding Unit}

\section{Branch Prediction}



\clearpage

\part{Results}

\subsection{Simulation}

The behaviour of the impementation was tested by writing small programs that would try to provoke certain errors and write debug data to the registers. They were then simulated, and the resulting data in the registers was compared to the desired data.

Alu operations, memory operations, forwarding, jumping, branching and flushing were all tested and proved to work correctly.

\subsection{Timing Simulation}

\begin{figure}[ht]
    \centering
    \includegraphics[scale=0.5]{figures/TimingSimulation.png}
    \caption{Timing diagram for critical path} 
    \label{fig:timing}
\end{figure}

Figure \ref{fig:timing} is a timing diagram showing the output from the ALU, which is the end of the critical path that starts at the ex/mem pipeline register and goes through the bottom link mux, the forwarding unit, the forwarding mux, and into the ALU.
The time from a rising clock edge to a the output is stable is shown to be $13.230$ ns, which translates to a maximum clock speed of about $75$ Mhz.
This is very close to the max clock speed of $77.442$ MHz reported by the synthesis tool.

\subsection{Verification}

By following the procedure in the compendium \cite[p.47]{lab-compendium} and using the supplied files, the design was added to the MicroBlaze framework and compiled to a bit file.


\begin{figure}[ht]
    \centering
    \includegraphics[scale=0.5]{figures/AVNET.png}
    \caption{FPGA programmed successfully} 
    \label{fig:avprog}
\end{figure}

\todo{stuff}



\clearpage

\part{Discussion}

\subsection{Warnings}

\subsection{Assembler}

\subsection{Further optimization}

\subsection{NOP}

Since the processor has no direct ability to halt the output from the memory blocks, a side effect of sending the output from the instruction memory directly through the first pipeline register is that the processor can only be reliably stopped on nops.
If not, the first instruction will be sent through the pipeline twice.

There are a couple solutions for this.
The first is to rewrite the supplied framework to allow the processor to halt the memory output, which creates a great deal of extra work and might cause bugs in other parts of the framework.
The second is to reset the program counter and flush the pipeline when the processor is stopped, which means that the program can not be resumed.

Since each solution has its own side effects, we decided to simply start all programs with a nop instruction.



\clearpage

\bibliography{bibtexlibz}{}
\bibliographystyle{plain}
\nocite{*}
All internet resources were checked on \today.
\end{document}
