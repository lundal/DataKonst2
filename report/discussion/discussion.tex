\subsection{Warnings}

There are quite a few warnings that pop up during syntezation of the project, but they are not the result of errors or mistakes in the processor implementation.
All come from the supplied framework, and most complain about memory address lines that are not connected, as the framework uses block ram which takes only 8-bit addresses.

\subsection{Assembler}

\subsection{Further optimization}

\subsection{NOP}

Since the processor has no direct ability to halt the output from the memory blocks, a side effect of sending the output from the instruction memory directly through the first pipeline register is that the processor can only be reliably stopped on nops.
If not, the first instruction will be sent through the pipeline twice.

There are a couple solutions for this.
The first is to rewrite the supplied framework to allow the processor to halt the memory output, which creates a great deal of extra work and might cause bugs in other parts of the framework.
The second is to reset the program counter and flush the pipeline when the processor is stopped, which means that the program can not be resumed.

Since each solution has its own side effects, we decided to simply start all programs with a nop instruction.

